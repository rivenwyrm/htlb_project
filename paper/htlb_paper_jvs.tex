\documentclass{article}

\title{HugeTLB Allocation Improvement to Reduce Post-Boot Failures}
\date{2016-02}
\author{Joshua Nicolas Verburg-Sachs}

\begin{document}
	\pagenumbering{gobble}
	\maketitle
	\newpage
	\pagenumbering{arabic}
	\section{Preamble}
		This research was inspired by many frustrated attempts to utilize the excellent huge page facilities of the Linux kernel during runtime. Huge pages are a mechanism whereby the Linux kernel can allocate memory not in 4kb pages (or whatever your distribution uses) but page sizes ranging from many megabytes to a gigabyte or more (though we touch only on the megabyte sized pages in this paper). This gives a variety of advantages, one of the most important being a reduction in the number of page entries that the kernel must manage for your executable. This can, for example, significantly increase the speed of large mmaps and munmaps.

Most of the current use cases and documentation stipulate that you should only attempt to allocate huge pages immediately after boot, in order to avoid (very common) failures due to memory fragmentation and memory use by other executables. However this is not always a convenient option and in some situations (such as real time systems with uptime requirements) it is basically untenable.

Therefore, this paper will describe a method whereby the Linux kernel can be improved to more robustly allocate huge pages during runtime, even when other applications have consumed a significant amount of available memory.

\end{document}